% From Approach

    \begin{equation}
    \label{eq:Start}
	E(z^{-1})=F(z^{-1})A(z^{-1})+z^{-d}G(z^{-1})
	\end{equation}
by matching the coefficients of $z^{-i}$ for both $F(z^{-1})$ and $G(z^{-1})$ with known polynomials $E(z^{-1})$ and $A(z^{-1})$ from \eqref{eq:E_q} and \eqref{eq:A_q}.
Now, we express \eqref{eq:ARMA_equation_revise} as,
	\begin{equation}
	\label{eq:ARMA_equation_revise2}
	E(z^{-1})y_i(k+d)={\alpha}z^{-1}y_i(k) + {\beta}z^{-1}u_i(k)
	\end{equation}
with the polynomials,
	\begin{equation}
	\alpha(z^{-1})=G(z^{-1})=\alpha_0+\alpha_1z^{-1}+ \dots +\alpha_{p-1}z^{-p+1}
	\end{equation}
	\begin{align}
	\begin{split}
	\beta( z^{-1})= F(z^{-1})&B^{'}(z^{-1})=\beta_0+\beta_1z^{-1} \\
	& + \dots +\beta_{q+d-1}z^{-q-d+1}, \beta_0\neq0
	\end{split}
	\end{align}
The right side of equation \eqref{eq:ARMA_equation_revise2} can be expressed by para-metrization in the form,
    \begin{equation}
	E(z^{-1})y_i(k+d)=\bm{\theta}_0(k)\bm{\phi}(k)
	\end{equation}
with the unknown vector $\bm{\theta}_0$ and known signal vector $\bm{\phi}(k)$:
    \begin{equation}
	\bm{\theta}_0=(\alpha_0, \dots ,\alpha_{p-1},\beta_0, \dots ,\beta_{q+d-1})^T \in \R^{p+q+d}
	\end{equation}
	\begin{align}
	\begin{split}
	\bm{\phi}(k)&=(y_i(k), \dots ,y_i(k-p+1),u_i(k), \dots , \\
	& u_i(k-q-d+1))^T \in \R^{p+q+d}
	\end{split}
	\end{align}
Since we want the system to have the same characteristics as the reference model \eqref{eq:reference model_q}, we need to estimate the true parameter vector $\bm{\theta}_0$ with the estimated parameter vector $\bm{\theta}^T(k)$. 
    \begin{equation}
    \bm{\theta}(k)=(\theta_1(k), \dots ,\theta_p(k),\theta_{p+1}(k), \dots ,\theta_{p+q+d}(k))^T
	\end{equation}
We update the estimate $\bm{\theta}^T(k)$ of the true parameter vector $\bm{\theta}_0$ using a \textit{Modified Projection Algorithm}:
	\begin{equation}
	\label{eq:Modified_Proj_Algorithm}
	\bm{\theta}(k)=\bm{\theta}(k-1)+\frac{a(k)\bm{\phi}(k-d)e(k)}{c+\bm{\phi}^T(k-d)\bm{\phi}(k-d)}
	\end{equation}
	\begin{equation}
	e(k)=E(z^{-1})y_i(k)-\theta^T(k-1)\bm{\phi}(k-d)
	\end{equation}
	\begin{align*}
	\varepsilon<a(k)<2-\varepsilon, 0,\varepsilon<1, c>0
	\end{align*}
	
	\NB{too many details ...cut this part to the bare minimum. This is already been done. It's not our contribution}
To follow the desired tracking output $y^*(k)$ from the reference model, the adaptive control input $u(k)$ is then calculated from the equation:
    \begin{equation}
    \label{eq:tracking_model}
	\bm{\theta}^T(k)\bm{\phi}(k)=E(z^{-1})y_i^*(k+d)
	\end{equation}
By rearranging \eqref{eq:tracking_model} we isolate $u(k)$ to calculate our next input signal for each subsystem:
	\begin{align}
	\label{eq:End}
	u_i(k)=\frac{1}{\theta_{p+1}(k)}&(-\theta_1(k)y_i(k)-\theta_2(k)y_i(k-1)  \nonumber \\
    -\dots-\theta_p(k)y_i(k&-p-1)-\theta_{p+2}(k)u_i(k-1)  \\
	-\theta_{p+3}(k)u_i(k-2)-& \dots - \theta_{p+q+d}(k)u_i(k-q-d+1) \nonumber \\
	+g&H(z^{-1})r_i(k))^T \nonumber
	\end{align}
    \begin{equation}
	\theta_{p+1}(k)\neq0 \nonumber
	\end{equation}
	
	
	
	
	%%%%%% SAVED FROM SYSTEM MODELING
	
	 The state space model's \eqref{eq:discritzed_SS_model} discrete-time transfer function in the $z$-domain is,

	\begin{align}
	\begin{split}
	\label{eq:transfer_function}
        G(z) & = \frac{y(z)}{u(z)} = \bm{C}(z\bm{I}-\bm{A}_d)^{-1}\bm{B}_d  \\
	& = \frac{b_0z^{p-d}+b_1z^{p-d-1} +...+b_qz^{p-d-q}}{z^{p}+a_1z^{p-1}+...+a_{p-1}z+a_p} 
	\end{split}
	\end{align}
with $b_0\ne{0}$, $d>0$, $p-d-q\geq{0}$, $p$ and $q$ are orders of the transfer function's denominator and numerator, and $d$ is the system delay. The difference equation describing the system is:
    \begin{align}
    \begin{split}
	\label{eq:difference_equation}
	&y(k+1)=a_1y(k)+\dots+a_py(k-p) \\
	+b_0&u(k-d+1)+\dots+b_qy(k-q-d+1)
	\end{split}
	\end{align}
To achieve desired performance, a reference model is chosen for the system in \eqref{eq:transfer_function} to have the same characteristics. The reference model is expressed as:


% From \eqref{eq:transfer_function}, the system can be expressed as:
%     \begin{align}
%     \begin{split}
%     \label{ARMA_equation}
% 	(1+&a_1z^{-1}+...+a_pz^{-p})y(k) \\
% 	&=z^{-d}(b_0+b_1z^{-1}+...+b_qz^{-q})u(k)
% 	\end{split}
% 	\end{align}
% Now, by defining from \eqref{ARMA_equation}, 
% 	\begin{equation}
% 	\label{eq:A_q}
% 	A(z^{-1})=(1+a_1z^{-1}+...+a_pz^{-p})
% 	\end{equation}
% 	\begin{equation}
% 	\label{eq:B_prime}
% 	B^{'}(z^{-1})=(b_0+b_1z^{-1}+...+b_qz^{-q})
% 	\end{equation}
% we can revise \eqref{ARMA_equation} to the system expressed as:
% 	\begin{equation}
% 	\label{eq:ARMA_equation_revise}
% 	A(z^{-1})y(k)=z^{-d}B^{'}(z^{-1})u(k)
% 	\end{equation}
	
% We design a reference model for the system's output $y(k)$ to track, with a given input reference signal $r(k)$. This reference model is expressed as,
% 	\begin{equation}
% 	\label{eq:reference model_q}
% 	E(z^{-1})y^*(k+d)=z^{-d}gH(z^{-1})r(k)
% 	\end{equation}



	\begin{equation}
	\label{eq:reference model_z}
	\frac{y^*(z)}{r(z)}=\frac{z^{-d}gH(z^{-1})}{E(z^{-1})}
	\end{equation}
where,
    \begin{equation}
    \label{eq:E_q}
	E(z^{-1})=1+e_1z^{-1}+ \dots +e_wz^{-w}
	\end{equation}
	\begin{equation}
	H(z^{-1})=1+h_1z^{-1}+ \dots +h_{w-1}z^{-w+1}
	\end{equation}
and the variable $g$ is a constant to ensure a 1:1 steady state ratio of the reference $r$ and tracking signal $y^*$. 

