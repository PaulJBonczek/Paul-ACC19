\begin{section}{System Models and Architecture}
	
\label{sec:modeling}
In this section, the system dynamical, noise, and disturbance models used for control and motion are described. We also introduce the adaptive control model that will be used later on in our approach \ref{sec:Res_adapt_control}.

% Bold letters only for vectors
% Approach

\subsection{Vehicle Model}
In this work we consider a differential-drive ground vehicle model \cite{nutaro2011building} whose continuous-time dynamics can be described as:
%    \begin{align}
%        \dot{v}=\begin{cases}
%            \frac{1}{m}(F_l+F_r-(B_s+B_r)v), & \text{if turning},\\
%            \frac{1}{m}(F_l+F_r-B_rv), & \text{if not turning},
%        \end{cases} \\
%        \dot{\omega}=\begin{cases}
%            \frac{1}{I_z}(\frac{B}{2}(F_l-F_r)-B_l\omega), &\text{if turning},\\
%            0, & \text{if not turning}, 
%        \end{cases}
%	\end{align}
%	\begin{align}
%        \dot{\theta_h}=\omega, \dot{x}=v\sin(\theta_h), \dot{y}=v\cos(\theta_h).
%	\end{align}

	\begin{equation}
    \begin{split}
    \label{eqn:plant}
    \dot{v}&=\left\{
    \begin{array}{ll}
    \frac{1}{m}(F_l+F_r-(B_s+B_r)v, & \hbox{if } turning\\
    \frac{1}{m}(F_l+F_r-B_rv, & \hbox{if } not~turning
    \end{array}
    \right.\\
    \dot{\omega}&=\left\{
    \begin{array}{ll}
    \frac{1}{I_z}(\frac{B}{2}(F_l-F_r)-B_l\omega, & \hbox{if } turning\\
    0, & \hbox{if } not~turning
    \end{array}
    \right.\\
    \end{split}
    \end{equation}
    \begin{align}
     \dot{\theta_h}=\omega,\text{ } \dot{x}=v\sin(\theta_h),\text{ } \dot{y}=v\cos(\theta_h).
	\end{align}
	
with position described by $x$ and $y$ coordinates, $\theta_h$ represents the vehicle's heading angle measured from the positive $x$ axis, and $v$ is the speed in this direction. $F_l$ and $F_r$ describe the left and right forces from the wheels, while $B_r$ is the mechanical resistance of the wheels to rolling. If the assumption is that the wheels do not slip.
This presents a high-level model of the vehicle, describing only the motion equations. Both forces, $F_l$ and $F_r$, are considered as inputs to the system, are derived from the vehicle's electromotors and are affected by the motors, gearbox, and wheels. All of the resistive parameters, mass, and moment of inertia $I_z$ are able to change to unknown future values as damage, system degradation, and environmental changes occur. The above system model may be expressed in a continuous state space form:
    \begin{equation}
	\dot{\bm{x}}(t) = \bm{A}(t)\bm{x}(t) + \bm{B}(t)\bm{u}(t)
	\end{equation}

\subsection{Adaptive Control Model}
In this work we leverage the \textit{Characteristic Model Based All-coefficient Adaptive Control} approach presented in \cite{4106038} in which the objective is to compute an input to have the system follow a reference model given changing dynamics. Here we summarize the model which will be used later in Section \ref{sec:Res_adapt_control} to adapt the system under uncertainties and to detect attacks. The dynamically changing model is linearized to a defined set point from initially known parameters within the $\bm{A}$ and $\bm{B}$ matrices:
	\begin{equation}
	\dot{\bm{x}}(t) = \bm{A}\bm{x}(t) + \bm{B}\bm{u}(t)
	\end{equation}
Since the adaptive control algorithm used is based on a discrete-time state space model, the continuous-time model needs to be discritized. With a suitable sampling time $t_s$, the linearized discrete-time model is written as,
    \begin{equation}
	\bm{x}(k+1) = \bm{A}_d\bm{x}(k) + \bm{B}_d\bm{u}(k)
	\end{equation}

with a discrete-time transfer function:
	\begin{align}
        G(z) & = \frac{y(z)}{u(z)} = \bm{C}(z\bm{I}-\bm{A}_d)^{-1}\bm{B}_d  \\
	& = \frac{b_0z^{n-d}+b_1z^{n-d-1} +...+b_mz^{n-d-m}}{z^{n}+a_1z^{n-1}+...+a_{n-1}z+a_n} \nonumber
	\end{align}
	\begin{equation}
	\text{where } b_0\ne{0}, d>0, \text{ and } n-d-m\geq{0} \nonumber
	\end{equation}
Using an Autoregressive Moving Average (ARMA) model and defining the operators $q^{-1}$ (delay) and $q$ (advance),
	\begin{equation}
	q^{-1}x(k) = x(k-1), qx(k) = x(k+1)
	\end{equation}
the system is now expressed as:
    \begin{align}
	(1+&a_1q^{-1}+...+a_nq^{-n})y(k) \nonumber \\
	&=q^{-d}(b_0+b_1q^{-1}+...+b_mq^{-m})u(k)
	\end{align}
where:
	\begin{equation}
	A(q^{-1})=(1+a_1q^{-1}+...+a_nq^{-n}) \nonumber
	\end{equation}
	\begin{equation}
	B^{'}(q^{-1})=(b_0+b_1q^{-1}+...+b_mq^{-m}) \nonumber
	\end{equation}
to form the revised system expression as:
	\begin{equation}
	A(q^{-1})y(k)=q^{-d}B^{'}(q^{-1})u(k)
	\end{equation}
We want to design a model reference for the system to follow while tracking a given signal. The system will track the given reference signal $r(k)$ at a rate determined by the model. This reference model is expressed as,
	\begin{equation}
	E(q^{-1})y^*(k+d)=q^{-d}gH(q^{-1})r(k)
	\end{equation}
to form a model reference transfer function in the form:
	\begin{equation}
	\frac{y^*(z)}{r(z)}=\frac{z^{-d}gH(z^{-1})}{E(z^{-1})}
	\end{equation}


 \subsection{Noise and Disturbance Model}
When a ground vehicle is in motion, there are many factors that may cause it to behave differently. These factors can include sensor or actuator noises and actuator failure, as well as external influences like wind disturbance, and a changing ground surface. The uncertainties from actuator noise are due to mechanical uncertainties of motors and gears, which are represented by $ \bm{\eta}_a \in \mathcal{N}(0,\sigma_a) $, while any disturbances are described as $\bm{d}(k)$. The output uncertainty due to measurement noise and sensor attacks is described by $ \bm{\eta}_s \in \mathcal{N}(0,\sigma_s) $ and $\bm{\xi}_a(k)$, respectively. All noises are assumed to have a zero mean Gaussian distribution. The overall discretized system model can be in the form:
    \begin{equation}
	\bm{x}(k+1)=\bm{A}(k)\bm{x}(k)+\bm{B}(k)\bm{u}(k)+\bm{\eta}_a(k)+\bm{d}(k)
	\end{equation}
	\begin{equation}
	\bm{y}(k)=\bm{C}\bm{x}(k)+\bm{\eta}_s(k)+\bm{\xi}_a(k)
	\end{equation}


\end{section}