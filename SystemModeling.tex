\begin{section}{System Models and Architecture}
	
\label{sec:modeling}
In this section, the system dynamical, noise, and disturbance models used for control and motion planning are described. We also introduce the adaptive control model that will be used later on in our approach in Section \ref{sec:Res_adapt_control}.

% Bold letters only for vectors
% Approach

\subsection{Vehicle Model}
In this work we consider a differential-drive ground vehicle model \cite{nutaro2011building} whose continuous-time dynamics can be described as:
%    \begin{align}
%        \dot{v}=\begin{cases}
%            \frac{1}{m}(F_l+F_r-(B_s+B_r)v), & \text{if turning},\\
%            \frac{1}{m}(F_l+F_r-B_rv), & \text{if not turning},
%        \end{cases} \\
%        \dot{\omega}=\begin{cases}
%            \frac{1}{I_z}(\frac{B}{2}(F_l-F_r)-B_l\omega), &\text{if turning},\\
%            0, & \text{if not turning}, 
%        \end{cases}
%	\end{align}
%	\begin{align}
%        \dot{\theta_h}=\omega, \dot{x}=v\sin(\theta_h), \dot{y}=v\cos(\theta_h).
%	\end{align}
\vspace{-5pt}
	\begin{equation}
    \begin{split}
    \label{eqn:plant}
    \dot{v}&=\left\{
    \begin{array}{ll}
    \frac{1}{m}(F_l+F_r-(B_s+B_r)v, & \hbox{if } turning\\
    \frac{1}{m}(F_l+F_r-B_rv, & \hbox{if } not~turning
    \end{array}
    \right.\\
    \dot{\omega}&=\left\{
    \begin{array}{ll}
    \frac{1}{I_z}(\frac{B}{2}(F_l-F_r)-B_l\omega, \text{      } & \text{     if } turning\\
    0, \text{      } & \text{     if } not~turning
    \end{array}
    \right.\\
    \end{split}
    \end{equation}
    \vspace{-10pt}
    \begin{align}
    \label{eq:plant2}
     \dot{\theta_h}=\omega,\text{ } \dot{x}=v\sin(\theta_h),\text{ } \dot{y}=v\cos(\theta_h).
	\end{align}
in which $x$ and $y$ coordinates represent position, $\theta_h$ represents the vehicle's heading
% angle measured from the positive $x$ axis, 
 and $v$ is the velocity. $F_l$ and $F_r$ describe the left and right forces from the wheels, while $B_r$ is the mechanical resistance of the wheels to rolling and $B_s$ is the resistance due to turning.
% If the assumption is that the wheels do not slip.
\eqref{eqn:plant} and \eqref{eq:plant2} present a high-level model of the vehicle, describing only the motion equations. Both forces, $F_l$ and $F_r$, are considered as inputs to the system, are derived from the motors, gearbox, and wheels. All of the resistive parameters, mass, and moment of inertia $I_z$ can to change to unknown future values as damage, system degradation, and environmental changes occur. The above system model may be rewritten in a time-varying continuous-time state space form,
    \begin{equation}
    \begin{split}
	\dot{\bm{x}}(t) = \bm{A}(t)&\bm{x}(t) + \bm{B}(t)\bm{u}(t) \\
	\bm{y}(t) &= \bm{C}\bm{x}(t)
	\end{split}
	\end{equation}
with the state matrix $\bm{A}(t) \in \R^{n\times n}$, the input matrix $\bm{B}(t) \in \R^{n\times m}$, output matrix $\bm{C} \in \R^{s\times n}$, state vector $\bm{x}(t) \in \R^{n}$, system input $\bm{u}(t) \in \R^{m}$, and output vector $\bm{y}(t) \in \R^{s}$. $\bm{A}(t)$, $\bm{B}(t)$, $\bm{x}(t)$, and $\bm{u}(t)$ are given by: 

%    \begin{equation}
%	\bm{A}_v(t)=\begin{bmatrix} 1 & 0 & cos(\theta_h) \\ 0 & 1 & sin(\theta_h) \\ 0 & 0 & -(B_s+B_r) \end{bmatrix},
%	\bm{B}_v(t)=\begin{bmatrix} 0 \\ 0 \\ \frac{1}{m} \end{bmatrix}, \nonumber
%	\end{equation}
%	\begin{equation}
%     \bm{x}_v(t)=\begin{bmatrix} x \\ y \\ v \end{bmatrix}, \bm{u}_v(t)= F_l+F_r 
%	\end{equation}
%The angle model for steering is written as:
%	\begin{equation}
%	\bm{A}_{\theta}(t)=\begin{bmatrix} 0 & 1 \\ 0 & -B_l \end{bmatrix} \nonumber
%	\bm{B}_{\theta}(t)=\begin{bmatrix} 0 \\ \frac{B}{2I_z} \end{bmatrix},
%	\end{equation}
%    \begin{equation}
%     \bm{x}_{\theta}(t)=\begin{bmatrix} \theta_h \\ \dot{\theta_h} \end{bmatrix}, \bm{u}_{\theta}(t)=F_l-F_r 
%	\end{equation}
%For either system, a general state space model is written in the form,
%    \begin{equation}
%	\dot{\bm{x}}(t) = \bm{A}(t)\bm{x}(t) + \bm{B}(t)\bm{u}(t)
%	\end{equation}

    \begin{equation}
	\bm{A}(t)=\begin{bmatrix} 0 & 0 & cos(\theta_h) & 0 & 0\\ 0 & 0 & sin(\theta_h) & 0 & 0 \\ 0 & 0 & -(B_s+B_r) & 0 & 0 \\ 0 & 0 & 0 & 0 & 1 \\ 0 & 0 & 0 & 0 & -B_l \end{bmatrix} \nonumber
	\end{equation}
	\begin{equation}
	\bm{B}(t)=\begin{bmatrix} 0 & 0 \\ 0 & 0 \\ \frac{1}{m} & 0 \\ 0 & 0 \\ 0 & \frac{B}{2I_z} \end{bmatrix}, 
     \bm{x}(t)=\begin{bmatrix} x \\ y \\ v \\ \theta_h \\ \dot{\theta_h} \end{bmatrix}, \bm{u}(t)= \begin{bmatrix} F_l+F_r \\ F_l-F_r  \end{bmatrix}  \nonumber
	\end{equation}

The model is initially defined by known parameters within the $\bm{A}$ and $\bm{B}$ matrices. Since the adaptive control algorithm is based on a discrete-time state space model, the continuous-time model needs to be discretized. With a suitable sampling time $t_s$, the linearized discrete-time model is written as,
    \begin{equation}
    \begin{split}
    \label{eq:discritzed_SS_model}
	\bm{x}(k+1) &= \bm{A}_d\bm{x}(k) + \bm{B}_d\bm{u}(k) \\
	\bm{y}&(k)=\bm{C}\bm{x}(k)
	\end{split}
	\end{equation}
with $\bm{y}(k) \in \R^{s}$ described as the output measurement vector, $\bm{C} \in \R^{s\times n}$ is defined as the output matrix, while $\bm{A}_d \in \R^{n\times n}$ and $\bm{B}_d \in \R^{n\times m}$.

%The uncertainties from actuator noise $\bm{\eta}_a \in \mathcal{N}(0,\sigma_a)$ are due to mechanical uncertainties of motors and gears,
 \subsection{Noise, Disturbance, and Attack Models}
When a ground vehicle is in motion, there are many factors that may cause it to behave differently. These factors can include sensor or actuator noises and actuator failure, as well as external influences like wind disturbance, and changing ground surfaces. All noises are assumed to have a zero mean Normal distribution $\mathcal{N}$ and are assumed to be a truncated with noise bounded at $N_{\sigma}$ standard deviations. The uncertainty due to measurement noise is described as $\bm{\eta}_s \in \mathcal{N}(0,\sigma_s)$, while disturbances $\bm{d}(k)$ are due to external factors like changes in surface material. In this work we consider that an attacker can observe and manipulate some of the sensor measurements by altering their value at any time. The sensor attack vector $\bm{\xi}$ is assumed to be unknown. Putting everything together, the overall discretized system model accounting for noise, disturbance, and attack effects can be described in in the following form:
    \begin{equation}
    \begin{split}
    \label{eq:discrete_SS}
	\bm{x}(k+1)=&\bm{A}_d(k)\bm{x}(k)+\bm{B}_d(k)\bm{u}(k)+\bm{d}(k) \\
	\bm{y}(k)&=\bm{C}\bm{x}(k)+\bm{\eta}_s(k)+\bm{\xi}(k)
	\end{split}
	\end{equation}

% λk is the attack vector in which each term represents attack of magnitude kλk,ik. The attack vector is assumed to be arbitrary, i.e., we assume no prior knowledge of statistical properties of bounds on the values of the attack vector.


\subsection{Adaptive Control Model}
The model in \eqref{eq:discrete_SS} can be transformed in a discrete-time transfer function form which is a more usable form for the resilient adaptive control approach later in \ref{sec:Res_adapt_control}. To solve Problem 1 and detect cyber-attacks while adapting the system, we leverage the CMBAC adaptive control approach presented in \cite{4106038}, in which the objective is to compute an input to have the system follow a reference model given changing dynamics (i.e. dynamic $\bm{A_d}$ and  $\bm{B_d}$). Here we summarize the model which will be used later in Section \ref{sec:Res_adapt_control} to adapt the system under uncertainties and to detect attacks. The state space model in a \eqref{eq:discrete_SS} transformed in discrete-time transfer function, in single-input single-output (SISO) form, within the $z$-domain becomes:
	\begin{align}
	\begin{split}
	\label{eq:transfer_function}
        G(z) & = \frac{y(z)}{u(z)} = \bm{C}(z\bm{I}-\bm{A}_d)^{-1}\bm{B}_d  \\
	& = \frac{b_0z^{p-d}+b_1z^{p-d-1} +...+b_qz^{p-d-q}}{z^{p}+a_1z^{p-1}+...+a_{p-1}z+a_p} 
	\end{split}
	\end{align}
with $b_0\ne{0}$, $d>0$, $p-d-q\geq{0}$, $p$ and $q$ are orders of the denominator and numerator, and $d$ is the system delay. A reference model of desired characteristics is chosen for the system in \eqref{eq:transfer_function} to follow, which can be expressed as:

% The difference equation describing the system is:
%     \begin{align}
%     \begin{split}
% 	\label{eq:difference_equation}
% 	&y(k+1)=a_1y(k)+\dots+a_py(k-p) \\
% 	+b_0&u(k-d+1)+\dots+b_qu(k-q-d+1)
% 	\end{split}
% 	\end{align}

% From \eqref{eq:transfer_function}, the system can be expressed as:
%     \begin{align}
%     \begin{split}
%     \label{ARMA_equation}
% 	(1+&a_1z^{-1}+...+a_pz^{-p})y(k) \\
% 	&=z^{-d}(b_0+b_1z^{-1}+...+b_qz^{-q})u(k)
% 	\end{split}
% 	\end{align}
% Now, by defining from \eqref{ARMA_equation}, 
% 	\begin{equation}
% 	\label{eq:A_q}
% 	A(z^{-1})=(1+a_1z^{-1}+...+a_pz^{-p})
% 	\end{equation}
% 	\begin{equation}
% 	\label{eq:B_prime}
% 	B^{'}(z^{-1})=(b_0+b_1z^{-1}+...+b_qz^{-q})
% 	\end{equation}
% we can revise \eqref{ARMA_equation} to the system expressed as:
% 	\begin{equation}
% 	\label{eq:ARMA_equation_revise}
% 	A(z^{-1})y(k)=z^{-d}B^{'}(z^{-1})u(k)
% 	\end{equation}
	
% We design a reference model for the system's output $y(k)$ to track, with a given input reference signal $r(k)$. This reference model is expressed as,
% 	\begin{equation}
% 	\label{eq:reference model_q}
% 	E(z^{-1})y^*(k+d)=z^{-d}gH(z^{-1})r(k)
% 	\end{equation}



	\begin{equation}
	\label{eq:reference model_z}
	\frac{y^*(z)}{r(z)}=\frac{z^{-d}gH(z^{-1})}{E(z^{-1})}
	\end{equation}
% where,
%     \begin{equation}
%     \label{eq:E_q}
% 	E(z^{-1})=1+e_1z^{-1}+ \dots +e_wz^{-w}
% 	\end{equation}
% 	\begin{equation}
% 	H(z^{-1})=1+h_1z^{-1}+ \dots +h_{w-1}z^{-w+1}
% 	\end{equation}
where $g$ is a constant to ensure a 1:1 steady state ratio of the reference $r$ and tracking signal $y^*$. 



\end{section}