\begin{section}{System Models and Architecture}
	
\label{sec:modeling}
In this section, the system dynamical, noise, and disturbance models used for control and motion are described, as well as the adaptive control scheme and confidence interval model.

% Bold letters only for vectors
% Approach

\subsection{Vehicle Model}
To attain a dynamical model of the ground vehicle, a generalized differential-drive model may be used \cite{nutaro2011building}. The vehicle's position is described by its $x$ and $y$ coordinates, $\theta$ represents the vehicle's heading angle measured from the $x$ axis, and $v$ is the speed in the direction in this direction. $F_l$ and $F_r$ describe the left and right forces from the wheels, while $B_r$ is the mechanical resistance of the wheels to rolling. If the assumption is that the wheels do not slip, the continuous time dynamical model of the vehicle is
    \begin{align}
        \dot{v}=\begin{cases}
            \frac{1}{m}(F_l+F_r-(B_s+B_r)v), & \text{if turning},\\
            \frac{1}{m}(F_l+F_r-B_rv), & \text{if not turning},
        \end{cases}
    \end{align}
    \begin{align}
        \dot{\omega}=\begin{cases}
            \frac{1}{I_z}(\frac{B}{2}(F_l-F_r)-B_l\omega), &\text{if turning},\\
            0, & \text{if not turning},
        \end{cases}
	\end{align}
	\begin{align}
        \dot{\theta_h}=\omega, \dot{x}=v\sin(\theta_h), \dot{y}=v\cos(\theta_h).
	\end{align}
This presents a high-level model of the vehicle, describing only the motion equations. Both forces, $F_l$ and $F_r$, are considered as inputs to the system, are derived from the vehicle's electromotors and are affected by the motors, gearbox, and wheels. All of the resistive parameters, mass, and moment of inertia $I_z$ are able to change to unknown future values as damage, system degradation, and environmental changes occur. This changing dynamic model can be described in the discretized state space form:
    \begin{equation}
	\bm{x}(k+1) = \bm{A}(k)\bm{x}(k) + \bm{B}(k)\bm{u}(k)
	\end{equation}

 \subsection{Noise and Disturbance Model}
When a ground vehicle is in motion, there are many factors that may cause it to behave differently. These factors can include sensor or actuator noises and actuator failure, as well as external influences like wind disturbance, and a changing ground surface. The uncertainties from actuator noise are due to mechanical uncertainties of motors and gears, which are represented by $ \bm{\eta}_a \in \mathcal{N}(0,\sigma_a) $, while any disturbances are described as $\bm{d}(k)$. The output uncertainty due to measurement noise and sensor attacks is described by $ \bm{\eta}_s \in \mathcal{N}(0,\sigma_s) $ and $\bm{\xi}_a(k)$, respectively. All noises are assumed to have a zero mean Gaussian distribution. The overall discretized system model can be in the form:
    \begin{equation}
	\bm{x}(k+1)=\bm{A}(k)\bm{x}(k)+\bm{B}(k)\bm{u}(k)+\bm{\eta}_a(k)+\bm{d}(k)
	\end{equation}
	\begin{equation}
	\bm{y}(k)=\bm{C}\bm{x}(k)+\bm{\eta}_s(k)+\bm{\xi}_a(k)
	\end{equation}

\subsection{Adaptive Control Model}
The control algorithm used is characteristic polynomial adaptive control based on the following discrete-time state space model with known dimensions of $\bm{A}$ and $\bm{B}$. The dynamically changing model is linearized by choosing initially known state space parameters for a specific operating point:
	\begin{equation}
	\bm{x}(k+1) = \bm{A}\bm{x}(k) + \bm{B}\bm{u}(k)
	\end{equation}
with its discrete time transfer function:
	\begin{equation}
        G(z) = \frac{y(z)}{u(z)} = \bm{C}(z\bm{I}-\bm{A})^{-1}\bm{B} \nonumber \\
    \end{equation}
    \begin{equation}
	= \frac{b_0z^{n-d}+b_1z^{n-d-1} +...+b_mz^{n-d-m}}{z^{n}+a_1z^{n-1}+...+a_{n-1}z+a_n}
	\end{equation}
where $b_0\ne{0}$, $d>0$, and $n-d-m\geq{0}$\\
Using an ARMA model and defining the operators $q^{-1}$ (delay) and q (advance),
	\begin{equation}
	q^{-1}x(k) = x(k-1), qx(k) = x(k+1)
	\end{equation}
the system is now expressed as:
    \begin{align}
	(1+&a_1q^{-1}+...+a_nq^{-n})y(k) \nonumber \\
	&=q^{-d}(b_0+b_1q^{-1}+...+b_mq^{-m})u(k)
	\end{align}
where:
	\begin{equation}
	A(q^{-1})=(1+a_1q^{-1}+...+a_nq^{-n}) \nonumber
	\end{equation}
	\begin{equation}
	B^{'}(q^{-1})=(b_0+b_1q^{-1}+...+b_mq^{-m})u(k) \nonumber
	\end{equation}
to form the revised system expression as:
	\begin{equation}
	A(q^{-1})y(k)=q^{-d}B^{'}(q^{-1})u(k)
	\end{equation}
We want to design a model reference for the system to follow while tracking a given signal. The system will track the given reference signal $r(k)$ at a rate determined by the model. This reference model is expressed as,
	\begin{equation}
	E(q^{-1})y^*(k+d)=q^{-d}gH(q^{-1})r(k)
	\end{equation}
to form a model reference transfer function in the form:
	\begin{equation}
	\frac{y^*(z)}{r(z)}=\frac{z^{-d}gH(z^{-1})}{E(z^{-1})}
	\end{equation}

\end{section}