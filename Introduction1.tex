
\section{Introduction} \label{sec:introduction}

Today's autonomous vehicles are fitted with numerous on-board sensors and computers that make them suitable for many civilian and military applications. These new complex capabilities have transformed the uses of vehicles to improve our daily lives. With these new enhancements and capabilities, comes the risk of more security vulnerabilities to attacks and failures. An adversary now has a greater opportunity of performing malicious attacks on sensor systems to degrade performance or completely lead them to an accident. An example of a destructive malicious attack is when the vehicle's GPS signals are spoofed, leading the vehicle into an obstacle (CITE!).

Several different solutions have been presented to address this issue, from sensor redundancy to complex filtering techniques (CITE!!!). However, a solution for detecting malicious attacks on a system with an unknown model due to changes in dynamics or disturbances has yet to be proposed. This paper will introduce an adaptive control and planning technique (COME UP WITH A UNIQUE NAME?) to detect sensor attacks on a dynamically changing system that will adaptively operate safely along a trajectory in an obstacle filled environment with noisy sensor measurement data. Our goal is to able to distinguish between changing dynamics or system degradation and sensor attacks. Furthermore, the system needs to update the vehicle's motion along a trajectory dependent on the availability of uncompromised sensor noise profiles \NB{rephrase to emphasize that due to the change in sensor configuration, the system in not able to perform as designed and thus we need to take into account the limited sensing resources to plan and move the system}\NB{add more details about the proposed technique}
.

The contribution of this paper comes in two parts: 1) we design a technique that is able to detect sensor attacks on an unknown, dynamically changing system, 2) after the compromised sensor has been removed, we propose a motion replanning technique to guarantee vehicle safety (i.e., something bad will never happen) and liveness (i.e., something good will eventually happen) while navigating in an obstacle filled environment using limited sensors of varying noise profiles. 

%%%%%%%

%The first part will be talking about the problem at hand. What is the problem? What needs to be solved or improved? Explain what has been missing in past techniques (cite them all). For example, we'll be utilizing adaptive control to compensate for failures and dynamical changes.

%Introduce a motivation for the paper. Why am I doing this? Then discuss what I'm going to be proposing in this paper. What am I going to be contributing? Something that no one has done yet, obviously. 

%Adapting of systems with changes/variations in dynamics, sensor noise, and attacks while also adaptively changing the system's motion planning on a high level.

%Discuss and show a timeline of all actions that happen in the simulations.

\subsection{Related Work}
\label{sec:Related Work}

Similar work in this area of Robotics, Adaptive Control, Security, Estimation, Motion Planning, Localization. Should contain 15-20 references in here (in total for paper). Explain everything in detail! Compare what other papers have to what I'm proposing in this paper. Maybe paper "x" didn't consider this or that, but in this paper we considered this or that. To show an improvement in the work. 

Mention Nicola's previous work (maybe 2-3 different papers). Towards the end of the related work, just include a brief organization of the paper describing what material is in each section and where they are located (approach, results, etc.).

