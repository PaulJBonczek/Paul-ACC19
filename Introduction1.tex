
\section{Introduction} \label{sec:introduction}

Today's autonomous vehicles are fitted with numerous on-board sensors and computers that make them suitable for many civilian and military applications. Many complex capabilities like autonomous navigation, surveillance, mapping, and manipulations have transformed the uses of vehicles to improve our daily lives. 
%These new complex capabilities have transformed the uses of vehicles to improve our daily lives. 
With these new enhancements and capabilities, comes the risk of more security vulnerabilities to attacks and failures. An adversary has the opportunity of performing malicious attacks on sensor systems to degrade performance or completely lead the vehicle to an undesired state. For example, a vehicle could be hijacked into an undesired region by spoofing its on-board GPS as demonstrated in \cite{lee}.

Several different solutions have been presented to address sensor attacks, from sensor redundancy to complex filtering techniques \cite{fawzi2014secure,pasqualetti2013attack,6120187,6943080,7330811}. However, a solution for detecting malicious attacks on a system with an unknown model due to changes in dynamics or disturbances has yet to be proposed. This paper introduces a technique to detect sensor attacks on a dynamically changing system without losing control performance. We leverage an adaptive control algorithm to detect sensor under attacks and present an innovative adaptive planning framework to guarantee safety while the vehicle is navigating in an uncertain environment with changes in dynamics (e.g., due to a failure), disturbances (e.g., changes in surface material) and system's configuration (e.g., one or more sensors are removed after detecting that they are compromised). 
%along a trajectory with noisy sensor measurement data in an environment of undesired states. 
Our goal is to distinguish between malicious sensor attacks and dynamical changes or disturbances to the system. Once an attack is detected the system is operating without one or more sensors, hence the system is not able to perform as originally designed.  Thus, the vehicle motion needs to be adapted depending on the availability of the uncompromised sensors and their noise profiles. Adapting its motion ensures that the vehicle will not enter undesired states. 

The contribution of this paper comes in two parts: 1) we design a technique that is able to detect sensor attacks on an unknown, dynamically changing system, 2) after the compromised sensors have been removed, we propose a motion replanning technique to take into account the new sensor configuration and  guarantee autonomous vehicle safety (i.e., something bad will never happen) and liveness (i.e., something good will eventually happen) while navigating in an obstacle filled environment using limited sensors of various noise profiles. 
To the best of our knowledge this is the first work that considers control and planning adaptation in the context of cyber-physical systems (CPS) cybersecurity.

%%%%%%%

%The first part will be talking about the problem at hand. What is the problem? What needs to be solved or improved? Explain what has been missing in past techniques (cite them all). For example, we'll be utilizing adaptive control to compensate for failures and dynamical changes.

%Introduce a motivation for the paper. Why am I doing this? Then discuss what I'm going to be proposing in this paper. What am I going to be contributing? Something that no one has done yet, obviously. 

%Adapting of systems with changes/variations in dynamics, sensor noise, and attacks while also adaptively changing the system's motion planning on a high level.

%Discuss and show a timeline of all actions that happen in the simulations.

\subsection{Related Work}
\label{sec:Related Work}

The study of both security and safety of autonomous vehicles has gained a lot of popularity among the robotics community due to their potential military and civilian applications \NB{the subject is the study of security and safety. It's not the study of security and safety that has potential military and civilian applications}. Various problems researchers have been interested in solving are detection of malicious attacks and guaranteeing vehicle safety when uncertainties are present. Attackers could deliberately attempt to compromise the vehicle's integrity, control performance, and safety. \NB{the 2 previous sentences need to be rewritten} As demonstrated in \cite{lee} the ability to spoof GPS sensors to drive a yacht off course. \NB{you've already mentioned this. Just put this sentence in the intro. Also the sentence is incomplete}. Numerous control and model based techniques have been recently leveraged to prevent such future events from happening. For example, in \cite{6426811} authors leverage state feedback by changing system dynamics via pole placement.\NB{??? to do what?} In \cite{zhu2012resilient}, a receding-horizon control algorithm is used to defend against replay attacks. Attack detection using a known linear system model is presented in \cite{pasqualetti2013attack}. Sensor redundancy based on a linear system models are presented in \cite{fawzi2014secure,6943080,7330811} \NB{you need to add more details and a better summary of these papers!}. However, the aforementioned work and the majority of the literature on CPS cyber-security typically assume time invariant, static models where dynamics aren't changing due to degradation, damage, and environmental or external factors.

When uncertainties arise or changes in the system have occurred, the vehicle needs to alter its plan of motion. Work in reachability analysis \cite{8046382,7799325,5980268} leverage the system model and uncertainties to determine inputs to avoid potential obstacles. In \cite{5980508}, Random-Belief Trees are created to guarantee safety in uncertain environments for safe navigation. The work in \cite{6934041} calculate a Risky Area with the help of probability density functions to prevent vehicle collisions. \NB{you need to add more details and describe in 1, 2 lines their approach and results}


In this work, we build a framework to detect malicious sensor attacks on a dynamically changing system to protect integrity of control and safety. To accomplish this goal, we use an adaptive control algorithm to ensure control performance is not degraded while the system is changing. More uncertainties arise with an unknown system, so we want to differentiate between sensor attacks and dynamical changes \NB{fix these sentences...they are really not formal}. \NB{here you need to discuss more in details the state of the art in adaptive control. You should write that we leverage adaptive control and then discuss a few different papers and the one that we use in this work} From \cite{tao2003adaptive,Goodwin1643720}, a  discrete-time adaptive control approach is utilized to help us achieve these control objectives.

Differing from previous work, we are able to detect sensor attacks on systems experiencing changing dynamics while maintaining control performance.

The rest of the paper is organized as follows: in Section \ref{sec:problem}, we formally define the problems at hand and Section \ref{sec:modeling}, we define all system, noise, and disturbance models. Section \ref{sec:approach}, we discuss the approach taken to attack detection and navigational safety and then validation with simulation and experiments in Section \ref{sec:simulation}. Lastly, we draw conclusions and discuss future work in Section \ref{sec:conclusion}.


