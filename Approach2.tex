
\begin{section}{Approach}
\label{sec:approach}

% INSERT FIGURE FOR THE ARCHITECTURE OF THE SYSTEM HERE
%\begin{figure}[ht!]
%	\centering
%	\includegraphics[width=0.48\textwidth]{figures/system_arch.png}
%	\caption{Architecture of our adaptive system with safety guarantees.}
%	\label{fig:arch}
%\end{figure}


\subsection{Detection}
% Create an block diagram showing the system architecture of the detection scheme from the adaptive controller.

% Talk about how adaptive control is used for detection

\subsection{Sensor Removal}

% Finish the last few lines and last equation on laptop then copy & paste (of this and previous subsection) to here in the morning 8/30

\subsection{Estimation}
The statistical technique of confidence intervals from (Reference to proper section) is the foundation of how this work solves estimation. Confidence intervals are used to guarantee with a confidence percentage that the true mean $\bar{x}$ lies within a $2D$ radial interval $\epsilon$, which is determined by a standard deviation $\sigma$ and the number of samples $N$. 

Assuming the data being used for estimation, regardless if it's raw or filtered data, is that it comes in the form $\mathcal{N}(0,\sigma_d)$ where the data population standard deviation $\sigma_d$ is known for any sensor combination. From [reference to confidence interval equation] we can find an interval of a determined confidence percentage that the vehicle is within that region to ensure safety. 

The assumption of a confidence interval is the true mean remains static on the given axis. This cannot be assumed in the case of position estimation of a navigating vehicle. A pseudo-static form of a confidence interval is made to compensate for translation of position $\bm{x}(k-l)$, where $l=1,2,\dots,N-1$. The $N-1$ number of previous position data points are represented as if they all were taken from the current position $\bm{x}(k)$ in time. (SHOW THIS IN A FIGURE FOR BETTER VISUAL UNDERSTANDING, either Matlab or Powerpoint). These samples $\bm{x}_p$, where $p=1,2,\dots,N$ form the set $\mathcal{\bm{S}}=\begin{bmatrix}\bm{x}_1,\bm{x}_2,\dots,\bm{x}_N \end{bmatrix}$. Further, we can break down the set into two vectors $\bm{s}_x=\begin{bmatrix} x_x_1,x_x_2,\dots,x_x_N \end{bmatrix}$ and $\bm{s}_y=\begin{bmatrix} x_y_1,x_y_2,\dots,x_y_N \end{bmatrix}$ which represent the $x$ and $y$ coordinate data, respectively. 

Translating the data coordinates into a pseudo-static form will create the set $\mathcal{\bm{S}}_t$, creating the two pseudo-static vectors $\bm{s}_{t_x}=\begin{bmatrix} x_{t_x_1},x_{t_x_2},\dots,x_{t_x_N} \end{bmatrix}$ and $\bm{s}_{t_y}=\begin{bmatrix} x_{t_y_1},x_{t_y_2},\dots,\bm{x}_{t_y_N} \end{bmatrix}$. Each element of these updated vectors are calculated as:
    \begin{equation}
	x_{t_x_w} = x_x(k-w+1)+\sum_{j=1}^w v(k-j)\cos{\theta(k-j)T_s}
	\end{equation}
	\begin{equation}
	x_{t_y_w} = x_y(k-w+1)+\sum_{j=1}^w v(k-j)\sin{\theta(k-j)T_s}
	\end{equation}
The mean of each vector is written as $\bar{s}_x$ and $\bar{s}_y$ is used for the calculation of the confidence interval.

Creating a pseudo-static case with these past position data samples introduces higher uncertainty. Uncertainties regarding the actuator $\sigma_a$, current velocity sensor $\sigma_v$, and current heading angle sensor $\sigma_h$ need to be accounted for. To guarantee the true position of the vehicle is within the estimation radial bounds $\epsilon$, the maximum error in position due to the actuator and sensor noises over the furthest sample in time in $\mathcal{\bm{S}}$ is calculated as:
    \begin{equation}
	DISCUSS THIS
	\end{equation}
Doing this will protect the integrity of the confidence interval percentage, ensuring the vehicle is within the estimation bounds.


\subsection{Adaptive Motion Planning}



% INSERT ALGORITHM FOR REPLANNING VELOCITY
%\begin{algorithm}
%   \caption{Adaptive Motion for Safety Guarantee} 
%   \label{alg:adapt_motion} 
%    \begin{algorithmic}[1]
	
%	\State Known Information N, \sigma, z^{*}
%	\State Collect all previous N positions, velocities, and angles
%	\State Calculate Confidence interval for psuedo-static case
%	\State Calculate the initial reachable tube $ R(\boldsymbol{x}_0, \bm{u}(t), t \in [t_s, t_s+T] ) $
%	\State Calculate $ t_{s+1} $    

%	\end{algorithmic}
%\end{algorithm}



\end{section} 