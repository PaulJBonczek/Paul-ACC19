\begin{section}{Problem Statement}
	
\label{sec:problem}

In this work we consider autonomous vehicle operations in cluttered environments under different uncertainties including noise, failures, and cyber-attacks on its sensors. 
We are interested in finding a strategy to detect sensor attacks in an unknown, dynamically changing system and guarantee that the vehicle is always able to navigate safely in the environment. Specifically, a ground vehicle has an objective to navigate to one or more goals $g_i$ with $ i = \{1, 2, \dots, N_g\}$, $N_g \in \N$ positioned at $\bm{p}_{g_i}={\begin{bmatrix} x,y \end{bmatrix}}^T$, while guaranteeing safety among uncertainties within an environment, failures, system aging, noise, and possible cyber-attacks.


% True state of system and estimate is within the noise of the system x-x_hat < some norm
% be able to distinguish between dynamical changes and attacks and also adapt the motion planning to safety travel in the environment


% 1st section of adaptive control (approach) talk more about detection
% What kind of attacks can I detect?

% Lemmas to show something
Our first objective is to detect sensor attacks within a dynamically unknown system and then remove any compromised sensor to maintain stable and safe control of the vehicle through the remaining of its mission. Formally:

\begin{problem} 
\label{problem1} {\textbf{Attack Detection and Sensor Reconfiguration:}} 
 Given the system dynamics as a function of its state $ \bm{x} $, its input $ \bm{u}$, and disturbances $ \bm{d} $,
	\begin{equation}
		\bm{x}(k+1) = f(\bm{x}(k), \bm{u}(k), \bm{d}(k))
	\end{equation}
we want to find a policy $\mathcal{P}$ to:
\begin{enumerate}
	\item differentiate between malicious sensor attacks and changing dynamics or failures; 
	\item remove the compromised sensors to restore system integrity. %for all future time instances $k+h$. 
\end{enumerate}
all while maintaining the adaptive system's control performance. Solving this problem will allow the autonomous vehicle to safely navigate during dynamical changes while preventing an attacker from having an undesirable effect on the system. In this work we assume that multiple sensors are available to performance a certain operation. 
% needs to be taken into account during replanning. Thus we need to replan to consider that we have fewer sensors.
\end{problem}
Solving Problem 1 removes the compromised sensors, however leaving the system with fewer sensors as designed. Therefore, during replanning it is necessary to consider these changes in system's configuration. Formally:
	
\begin{problem} \label{problem2} {\textbf{Adaptive Motion Planning:}}
With Problem 1 solved, we want to find a policy to operate safely given a few number of sensors than originally designed. We want to adapt motion in such a way that the vehicle never enters a known undesired region,
	\begin{equation}
		\lVert {\bm{p}(k)-\bm{p}_{r_i}} \rVert >\Delta(k),  i \in \begin{bmatrix} 1,N_r \end{bmatrix}
	\end{equation}
where $\bm{p}(k)={\begin{bmatrix} x,y \end{bmatrix}}^T$ is the position of the vehicle, $\bm{p}_{r_i}$ is the positions of the ${i}^{th}$ regions to avoid in the $x-y$ plane, $\Delta(k)$ is a defined distance between the vehicle's position and an undesired region, and $N_r \in \N$ is the total number of undesired regions. Solving this problem allows the autonomous vehicle to adapt its motion according to the new sensor set, guaranteeing the vehicle never enters an undesired region.


	
\end{problem}

\end{section}