\begin{section}{Problem Statement}
	
\label{sec:problem}

In this work we consider autonomous vehicle operations in cluttered environments under different uncertainties including noise, failures, and cyber-attacks on its sensors. 
We are interested in finding a strategy to detect sensor attacks in an unknown, dynamically changing system and guarantee that the vehicle is always able to navigate safely in the environment. Specifically, a ground vehicle has an objective to navigate to one or more goals $g_i$ with $ i = 1, 2, \dots, N_g$, $N_g \in \N$ positioned at $\bm{p}_{g_i}={\begin{bmatrix} x,y \end{bmatrix}}^T$, while guaranteeing safety among uncertainties within an environment, failures, system aging, noise, and possible cyber-attacks.


% True state of system and estimate is within the noise of the system x-x_hat < some norm

Our first objective is to detect sensor attacks within a dynamically unknown system and then remove any compromised sensor to maintain stable and safe control of the vehicle through the remaining of its mission. Formally:

\begin{problem} 
\label{problem1} {\textbf{Attack Detection and Sensor Reconfiguration:}} 
 Given the system dynamics as a function of its state $ \bm{x} $ and input $ \bm{u}$,
	\begin{equation}
	\begin{split}
		\dot{\bm{x}}(t) &= f(\bm{x}(t), \bm{u}(t)) \\
		\bm{y}(t)&=h(\bm{x}(t))
    \end{split}
	\end{equation}
find a policy $\mathcal{P}$ to:
\begin{enumerate}
	\item differentiate between malicious sensor attacks and changing dynamics or failures; 
	\begin{equation}
	\begin{split}
	\label{eq:detect_attack}
	    \dot{\bm{x}}(t) &= f(\bm{x}(t), \bm{u}(t), \bm{d}(t)) \\
	    \bm{y}(t)&=h(\bm{x}(t), \bm{\xi}(t))
	\end{split}    
	\end{equation}
where $\bm{d}$ is any dynamical changes or disturbances to the system and $\bm{\xi}$ is an attack on a vector.
	\item remove the compromised sensor measurements to restore system integrity.
	\begin{equation}
	\label{eq:remove_sensors}
	    \bm{y}'=\bm{y} \setminus \bm{y}_a
	\end{equation}
\end{enumerate}
where $\bm{y}$ is the set of available measurements, $\bm{y}_0$ is the initial set of measurements, and $\bm{y}_a$ is the set of measurements under attack.


% needs to be taken into account during replanning. Thus we need to replan to consider that we have fewer sensors.
\end{problem}
Solving Problem 1 removes the compromised sensors, however leaving the system with fewer sensors as designed. Therefore, during replanning it is necessary to consider these changes in system's configuration. Formally:
	
\begin{problem} \label{problem2} {\textbf{Adaptive Motion Planning:}}
Given \eqref{eq:detect_attack} and \eqref{eq:remove_sensors} are satisfied, find a policy to adapt the motion planning strategy to guarantee safety, i.e. adapt velocity in such a way that the vehicle is guaranteed to never enter any undesired state $\bm{x}_o$:  
	\begin{equation}
		\lVert {\bm{x}(t)-\bm{x}_o} \rVert > 0
	\end{equation}
where $\bm{x}(t)$ is the state of the vehicle and $\bm{x}_o$ is the state of any known undesired region.
	\end{problem}

%Solving this problem will allow the autonomous vehicle to safely navigate during dynamical changes while preventing an attacker from having an undesirable effect on the system. In this work we assume that multiple sensors are available to performance a certain operation. 
In our specific problem, we look at the position of the vehicle in relation to undesired states like obstacles and unsafe regions to navigate. Solving this problem allows the autonomous vehicle to adapt its motion according to the new sensor set, guaranteeing the vehicle never enters an undesired region.
\end{section}