\begin{section}{Background}
	
\label{sec:model}
In this section, the system dynamical, noise, and disturbance models used for control and motion are described, as well as the adaptive control scheme used.

\subsection{Adaptive Control Model}
The control algorithm used is characteristic polynomial adaptive control based on the following discrete-time state space model,
	\begin{equation}
	\bm{x}(k+1) = A\bm{x}(k) + B\bm{u}(k)
	\end{equation}
with its input to output transfer function 
	\begin{align}
	& \bm{G}(z) = \frac{y(z)}{u(z)} = C(zI-A)^{-1}B = \nonumber \\ 
	& \frac{b_0z^{n-d}+b_1z^{n-d-1}+...+b_mz^{n-d-m}}{z^{n}+a_1z^{n-1}+...+a_{n-1}z+a_n}
	\end{align}
where $b_0\ne{0}$, $d>0$, and $n-d-m\geq{0}$
using an ARMA model, defining the operators $q^{-1}$ (delay) and q (advance),
	\begin{equation}
	q^{-1}x(k) = x(k-1), qx(k) = x(k+1)
	\end{equation}
the system is now expressed as:
	\begin{align}
	& (1+a_1q^{-1}+...+a_nq^{-n})y(k) \nonumber \\
	& =q^{-d}(b_0+b_1q^{-1}+...+b_mq^{-m})u(k)
	\end{align}
where
	\begin{equation}
	Aq^{-1}y(k)=q^{-d}B^{'}(q^{-1})u(k)
	\end{equation}
The objective of model reference adaptive control is to calculate an input signal u(k) such that the system tracks the signal $y^{*}(k)$ given the reference signal r(k). The reference model the system is trying to follow is expressed as
	\begin{equation}
	E(q^{-1})y^*(k+d)=q^{-d}gH(q^{-1})r(k)
	\end{equation}
The system is parametrized as:
	\begin{equation}
	E(q^{-1})y(k+d)={\alpha}q^{-1}y(k) + {\beta}q^{-1}u(k)=\theta_0^T\phi(k)
	\end{equation}
	\begin{equation}
	\alpha(q^{-1})=G(q^{-1})=\alpha_0+\alpha_1q^{-1}+ ... +\alpha_{n-1}q^{-n+1}
	\end{equation}
	\begin{align}
	& \beta(q^{-1})=F(q^{-1})\beta^{'}(q^{-1})=\beta_0+ \nonumber \\
	& \beta_1q^{-1}+ ... +\beta_{m+d-1}q^{-m-d+1}, \beta_0\neq0
	\end{align}
	\begin{equation}
	E(q^{-1})=F(q^{-1})A(q^{-1})+q^{-d}G(q^{-1})
	\end{equation}
	Assumptions for adaptive control: \\
	1) all zeros of $B^{'}(z^{-1})z^m$ are within $|z|<1$. \\
	2) the system delay $\bm{d}$ is known. \\
	3) $\bm{n}$ and $\bm{m}$ (or their upper bounds) are known. \\
Parametrization to express this system is:
	\begin{equation}
	\theta_0=(\alpha_0, ... ,\alpha_{n-1},\beta_0, ... ,\beta_{m+d-1})^T \in R^{n+m+d}
	\end{equation}
	\begin{align}
	& \phi(j)=(y(k), ... ,y(k-n+1),u(k), ... , \nonumber \\
	& u(k-m-d+1))^T \in R^{n+m+d}
	\end{align}
where $\theta_0$ is unknown and $\phi(k)$ is known.
	The adaptive control input u(k) is then calculated from the equation
	\begin{equation}
	\theta^T(k)\phi(k)=E(q^{-1})y^{*}(k+d)
	\end{equation}
with the equation for the input u(k) as
	\begin{align}
	& u(k)=\frac{1}{\theta_{n+1}(k)}(-\theta_1(k)y(k)-\theta_2(k)y(k-1)- ...  \nonumber \\
	& -\theta_n(k)y(k-n-1)-\theta_{n+2}(k)u(k-1) \nonumber \\
	& -\theta_{n+3}(k)u(k-2)- ... - \theta_{n+m+d}(k)u(k-m-d+1) \nonumber \\
	& +y^{*}(k+d))^T
	\end{align}
	\begin{equation}
	\theta(k)=(\theta_1(k), ... ,\theta_n(k),\theta_{n+1}(k), ... ,\theta_{n+m+d}(k))^T
	\end{equation}
where $\theta(k)$ is the estimate of the true parameter vector $\theta_0$, updated by:
	\begin{equation}
	\theta(k)=\theta(k-1)+\frac{a(k)\phi(k-d)e(k)}{c+\phi^T(k-d)\phi(k-d)}
	\end{equation}
	\begin{equation}
	e(k)=E(q^{-1})y(k)-\theta^T(k-1)\phi(k-d)
	\end{equation}
	\begin{align*}
	\varepsilon<a(k)<2-\varepsilon, 0,\varepsilon<1, c>0
	\end{align*}
To avoid division by 0, $\theta_{n+1}(k)\neq0$ is necessary.

 \subsection{Noise and Disturbance Model}
When a ground vehicle is in motion, there are many factors that may cause it to behave differently. These factors can include sensor or actuator noises and actuator failure, as well as external influences like wind disturbance, and a changing ground surface. The uncertainties from actuator noise are due to mechanical uncertainties of motors and gears, which are represented by $ e_a \in N(0,\sigma_a) $. The uncertainty due to measurement noise of the sensors is described by $ d_s \in N(0,\sigma_s) $. All noises are assumed to have a zero mean Gaussian distribution.

 \subsection{Confidence Intervals}

Confidence intervals are used to estimate the mean of a population of data in stochastic environments. Using this method, a guarantee can be made that the mean of the data has a specific percentage of confidence that it's within an interval. Assuming the knowledge of the confidence percentage $ \bm{z^{*}} $, sensor noise population standard deviation $ \bm{\sigma_p} $, the number of sensor data samples $ \bm{N} $, and the mean of the data samples $ \bm{\bar{x}}_{x,y} $ in the $\bm{x}$ and $\bm{y}$ direction, a confidence interval of a certain percentage can be calculated: 
 	\begin{equation}
		\bm{CI}_{x,y|N} = \bm{\bar{x}}_{x,y} + \bm{z^{*}}\frac{\bm{\sigma}_p}{\sqrt{\bm{N}}}
	\end{equation}
This is under the assumption that the mean is not changing over the previous $\bm{N}$ number of sampled data points used in the current confidence interval calculation. A detailed approach to we compensate for a translating system will be shown in (REF THE SECTION HERE).

























\end{section}