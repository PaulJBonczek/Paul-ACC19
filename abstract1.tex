\NB{title needs to be changed to add autonomous vehicles}
\NB{authors and departments part is not following the ACC format, unless they changed this year.}
\begin{abstract}

Modern autonomous vehicles rely on their on-board sensors to effectively control and navigate safely through uncertain environments. When attackers maliciously compromise these vital sensors, the vehicle's performance is inhibited which can potentially direct the vehicle to an undesired state. In this paper we focus on two main problems: i) how to detect sensor attacks when the vehicle's dynamics are unknown or changing, while still maintaining control performance and ii) how to adapt system's performance to guarantee safety once an attack is removed and the system is operating with limited resources (e.g., fewer sensors than planned). To solve these problems, we propose an adaptive control-based technique to detect and remove sensors under attack and a recovery procedure that takes into account the updated limited sensor resources to estimate the state of the system with a certain confidence and continue the operation safely. Finally, we validate the proposed approach with simulations and experiments on an autonomous ground vehicle performing a go-to-goal mission in a cluttered environment while experiencing sensor attacks and changes in dynamics and disturbances.

%and isolate them from the rest of the system
 
%guaranteeing vehicle navigation safety when the attacked sensor is removed and now relying on sensor(s) of different uncertainty by adapting its motion. To accomplish this, we propose a control architecture utilizing an adaptive controller capable of detecting sensor attacks and an estimation technique to ensure confidence in vehicle safety. 

%in which a ground vehicle is given a trajectory to reach a goal while
\end{abstract}