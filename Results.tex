
\begin{section}{Simulation Results}
\label{sec:simulation}

% Need to label states of the state vector in the modeling section [x y v \theta]

The case study investigated in this paper is with a ground vehicle experiencing the effects of sensor and process noise, dynamical changes, while also succumbing to wind disturbances and sensor attacks. The vehicle will be traveling along a pre-planned trajectory with assumed to be known obstacles throughout the environment. (Show a figure, or more, of the vehicle within the environment showing the path and obstacles). We consider a ground vehicle starting at an initial position facing the positive $x$ direction with zero velocity $x(0)=(0,0,0,0)^T$. The objective for the vehicle is to encircle a large object while following a desired trajectory with obstacles of varying distances from the path. During navigation, parameters within the fundamental matrix $\bm{A}(k)$ and input matrix $\bm{B}(k)$ may change by up to 50\%. It is assumed the maximum velocity of the vehicle is 3.0m/s and the desired reference velocity it wants to maintain is 2.5m/s. 

% In Figure Xa) the vehicle moving along the trajectory with no compromised sensors and far from obstacles. Velocity is unaltered due to the obstacles being well outside the uncertainty boundaries.

% Figure Xb) demonstrates the effects of the loss of a position sensor of low noise profile. The confidence interval has grown in size.

% Figure Xc) demonstrates the adaptive motion planner in action when obstacles come within the confidence and uncertainty boundary. Velocity is reduced to shrink both the confidence interval and uncertainty bounds.

% Figure Xd) Obstacles are again far away from the given trajectory, outside of the CI and uncertainty bounds. Velocity is restored to desired levels.

% Figure Y) displays all sensor measurements for velocity, capturing when the attack occurs and the consequent recovery. 

% Figure Z) displays all position measurements?

% Figure XX) Show convergence of signal to reference

\begin{figure}
\vspace{1pt}
\centering
\includegraphics[width=0.48\textwidth]{sim_environment.png}
\caption{The obstacle filled environment the vehicle is traveling through to reach a specified goal point.}
\label{fig:sim_env}
\end{figure}

\begin{figure}
\vspace{1pt}
\centering
\includegraphics[width=0.48\textwidth]{vel_t.png}
\caption{Vehicle velocity over time exhibiting an attack on a velocity sensor while the system detects and corrects.}
\label{fig:vel_t}
\end{figure}

\begin{figure}
\vspace{1pt}
\centering
\includegraphics[width=0.48\textwidth]{ang_t.png}
\caption{Vehicle heading angle over time exhibiting an attack on an angle sensor while the system detects and corrects.}
\label{fig:ang_t}
\end{figure}

\end{section}